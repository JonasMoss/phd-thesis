\documentclass[english]{article}

\begin{document}
Explain publication bias and p-hacking.
Explain that it is possible to model them.
It's still hard to model though.


Science is built on trust. Trust in researchers to honestly report their results, trust in reviewers to do their job properly, and trust in the 

The scientific literature is not representative of all studies done. 

The academic life and prestige of researchers depend on finding positive, interesting results. But most studies will usually not give up positive results lightly. Usually, the sample sizes too small to reliably detect positive effects. Moreover, the probability of a new scientific idea being true is minuscule. Academics have a strong incentive to engage in questionable research practices to camouflage their null-results as positive and interesting. Some of these questionable practices are statistical. If an effect is significant only for men, report this is as your finding. If it is significant only for lesbian women over the age of 50, report that as your finding. If finding a significant effect requires minor readjustments of the numbers, do it.

Knowing that researchers engage in question research and that publication bias is rampant, is there anything of value in the scientific literature? This thesis is about statistical methods that correct for these sources of bias. 

\end{document}