\documentclass[english]{article}

\begin{document}
* Publication bias and p-hacking is bad.
* But I know how to fix the problem!

* Explain publication bias.
* Explain p-hacking.
* By modelling the process of publication bias and p-hacking one can save some information from 

* 

Publication bias and p-hacking are the scourges of science, breaking its proud edifice to shambles.


Science is built on trust. Trust in researchers to honestly report their results, trust in reviewers to do their job properly, trust in the high standards being upheld. This trust is not always warranted. The academic life and prestige of researchers depend on finding positive, interesting results. But most studies will usually not give up positive results lightly. The sample sizes are too small to reliably detect positive effects. The probability of a new scientific idea being true is minuscule. Academics have a strong incentive to engage in questionable research practices to camouflage their null-results as positive and interesting. Some of these questionable practices are statistical. If an effect is significant only for men, report this is as your finding. If it is significant only for lesbian women over the age of 50, report that as your finding. If finding a significant effect requires minor readjustments of the numbers, do it. 



This thesis is about how you can correct

Publication bias and p-hacking are the scourges of science.

The scientific literature is not representative of all studies done. 

The academic life and prestige of researchers depend on finding positive, interesting results. But most studies will usually not give up positive results lightly. Usually, the sample sizes too small to reliably detect positive effects. Moreover, the probability of a new scientific idea being true is minuscule. Academics have a strong incentive to engage in questionable research practices to camouflage their null-results as positive and interesting. Some of these questionable practices are statistical. If an effect is significant only for men, report this is as your finding. If it is significant only for lesbian women over the age of 50, report that as your finding. If finding a significant effect requires minor readjustments of the numbers, do it.

Knowing that researchers engage in question research and that publication bias is rampant, is there anything of value in the scientific literature? This thesis is about statistical methods that correct for these sources of bias.  

\end{document}