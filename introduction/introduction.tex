\section{Introduction: The lay of the land}

What exactly are confidence intervals again, and can I expect them to behave well? Do people know what a hypothesis test is and how to interpret them? The section on statistical inference is about misunderstanding of statistical concepts and impossibility results -- situations when frequentists constructions such as confidence sets and hypothesis tests fail to be well-behaved. 

Psychology is undergoing a replication crisis, and has done so since about $2011$. That was the year when \textcite{Bem2011-vq} published his study of the paranormal in the top-tier \textit{Journal of Social and Personality Psychology} and \textcite{simmons_false-positive_2011} published their famous \textit{False-Positive Psychology} paper. The problem can be summed up like this: You cannot trust what psychological research. The replication crisis and meta-analysis section supplies the details.

What is intelligence? How do we measure it? How many personality traits are there, and how do they matter? Psychometrics is the science of psychological measurement. There are two kinds of psychometricians; the theoretical and practical. Theoretical psychometricians are just like statisticians, they deal with mathematics and programming. They publish in journals such as \textit{Psychometrika}, \textit{British Journals of Mathematical Psychology}, and \textit{Multivariate Behavioural Research}. These journals are specialized methodological journals, and allow for the use of mathematics. Practical psychometrians design measurement instruments and administer them. They publish both in specialized psychometric journals such \textit{Psychological Assessment}, and methodologically generalist psychology journals such as \textit{Emotion.} The section on psychometrics shows the main models of psychometrics, discusses some fundamental questions, and gives an intuition for what questions we are dealing with, with a special emphasis on reliability.

Whenever you hear someone say something like ``we let $\theta=1$ to make the problem identified,'' chances are you you are facing a tucked-away \textit{partial identification} problem. In the section on partial identification I give some intuition about this subject.

This thesis contains two \textit{software papers}, both published in the \href{https://joss.theoj.org/}{\textit{Journal of Open Source Software}}. This journal is solely focused on high-quality software, including solid documentation and tests, and the paper itself is not at the centre stage. The papers in the Journal of Open Source Software are short, as the journal exist in order to make software a citeable part of the scientific literature. Journals with similar agendas are the \textit{R Journal} and the \textit{Journal of Statistical Software}.