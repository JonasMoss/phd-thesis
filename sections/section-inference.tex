\section{Statistical inference}

Bayesians are in a privileged position vis-à-vis frequentists. Any problem
has a unique solution for a Bayesian provided he can answer three
questions \parencite[Chapter 2]{Robert2007-ri}:
\begin{enumerate}
\item "What is the structure of reality?" He needs a likelihood for the problem. 
\item "What do I know about reality?" He needs a prior for the parameters
of the likelihood.
\item "What do I want to do, or what to I want to know?" He needs a loss function
to quantify the consequences of choosing the wrong act.
\end{enumerate}
The Bayesian does not only claim that every problem has a unique solution.
His claim is much stronger. Given a likelihood, a prior, and a loss
function, it is \emph{irrational} to do anything else than minimizing
the expected loss, also known as the risk, defined as
\begin{equation}
\argmin_{a\in\mathcal{A}}E[l(X,a)\mid\textrm{data}].\label{eq:risk minimization}
\end{equation}
Frequentist seldom have this luxury. Sometimes solutions with desireable
features do not exists, sometimes it is hard to find candidate confidence
set, sometimes optimality criteria do not hold. Since frequentist
methods usually aren't unique, frequently based on intuition, and
can be based on severely conflicting criteria, they are famously called \emph{ad hoc
methods}. \textcite[p. xxiii]{Jaynes2003-ky} speaks for many Bayesians
in his diatribe against frequentist statistics:
\begin{quote}
{[}F{]}requentist methods provide no technical means to eliminate
nuisance parameters or to take prior information into account, no
way even to use all the information in the data when sufficient or
ancillary statistics do not exist. Lacking the necessary theoretical
principles, they force one to \textquoteleft choose a statistic\textquoteright{}
from intuition rather than from probability theory, and then to invent
ad hoc devices (such as unbiased estimators, confidence intervals,
tail-area significance tests) not contained in the rules of probability
theory.
\end{quote}
The case for Bayesianism is extremely strong \parencite{Berger1988-ji,Bernardo2009-wv}.
But the arguments hinge on the assumption that we know the likelihood,
prior, and loss. Is this realistic? We will consider them in turn. 

The nature of Bayesian inference sometimes requires us to specify
much more than a frequentist needs to. A Bayesian always needs to
specify a prior, and the difficulties involved in finding good priors
takes up a large chunk of the arguments of frequentists against Bayesians,
and has inspired much work in objective Bayes. Getting the benefits
of Bayesian analysis without a using priors has even been called ``holy
grail of statistical theory'' by \textcite{Efron2010-is}. 

But the Bayesian also has to specify a likelihood. In most frequentist
statistics we assume a known likelihood, but not always. In particular,
semi-parametric statistics often assumes no likelihood. For instance,
the \textcite{Cox1972-xd} model assumes only a partially known likelihood,
while the maximum score model of \textcite{Manski1975-gl} makes no likelihood
assumptions at all. Moreover, non-parametric methods, such as kernel
density estimation \parencite{Silverman1986-nt} do not involve likelihoods
at all. Curiously, the archetypal non-parametric estimator, the histogram,
can be framed as a maximum likelihood problem, complete with its own
AIC \parencite{Birge2006-nl}. But even in this case, no likelihoods are
actually assumed. The histogram is just a handy device to estimate
an unknown density. That said, most semi-parametric and non-parametric
problems can be formulated into similar non-parametric Bayes models.
For instance, the natural analogue of kernel density estimation is
the normal Dirichlet mixture model \parencite[Chapter 2.2]{Muller2015-xn}.

But there are problems where Bayesian estimation appears impossible
in principle, namely when the likelihood does not exist.
\begin{example}[{No likelihood, \textcite[p. 30]{Berger1988-ji}}]
\label{exa:no likelihood}Let $\{P_{\theta}\},\,\theta\in[0,1]$
be a parameterized class of probability distributions on $[0,1]$
defined by
\begin{equation}
P_{\theta}(A)=\frac{1}{2}[\lambda(A)+1_{A}(\theta)],\label{eq:no likelihood}
\end{equation}
for all $A$ in the Borel $\sigma$-algebra of $[0,1]$. Here $\lambda$
is the Lebesgue measure. Then $\{P_{\theta}\}$ is not dominated by
any $\sigma$-finite measure. To see why, assume $Q$ dominates $\{P_{\theta}\}$,
that is, $Q(A)\geq P_{\theta}(A)$ for all $\theta$ and all $A$.
In particular, $Q(\theta)\geq1$ for all $\theta\in[0,1]$. By $\sigma$-additivity,
$Q(A)$ is finite if and only if $A$ is finite, and since a countable
union if finite sets is countable, $Q$ is not $\sigma$-finite.
\end{example}

Since $\{P_{\theta}\},\,\theta\in[0,1]$ has no likelihood, the Bayesian
is lost. He might have a prior on $\theta$, for instance a $\textrm{Beta}(2,2)$,
but that does not help him. He might want to use the quadratic loss
function, but he can't. Perhaps he could ditch the idea of likelihoods
altogether and use a predetermined countable partitioning of $\Omega$
instead. But such a solution is most definitely \emph{ad hoc}, and
hard to reconcile with his prior on $\theta$. The only coherent way
seems to be the adoption of a prior with countable support, so that
$\{P_{\theta}\}$ becomes dominated; but then the estimator will be
inconsistent for all $\theta$ outside of that countable support.

On the other hand, the frequentist has no trouble dealing with this
example. There are reasonable estimators for $\theta$ when we observe
repeatedly from $P_{\theta}$. Most obviously, the estimator $\hat{\theta}=a$
if and only if $a$ has been observed at least twice. Confidence sets,
convergence rates, and other frequentist properties are trivially
calculated from this estimator.

An even stranger is case $P_{\theta}(A)=1_{\theta}(A)$. Here $\theta$
is uniquely identified from a single observation, but has no likelihood,
and no Bayesian analysis is possible.

These two examples don't look realistic at all, and might be brushed
aside as mere curiosities. A more practically serious problem appears
when a likelihood appears to exist, is hard to reason about, but is
somehow not needed for the problem.
\begin{example}[Robins--Ritov--Wasserman]
 This example due to \textcite{Robins2012-fr}. We will consider a sequence
of $n$ independent and identically distributed variables $(X_{i},Y_{i},Y_{i}^{*},R_{i})$,
where $X_{i}\in[0,1]$ is uniform,  $R_{i}$ is binary, $Y_{i}^{*}\in\{0,1\}$,
and $Y\in\{\textrm{NA},0,1\}$. Here $Y_i^{*}$ are unobserved; instead
of observing $Y_{i}^{*}$, we observe
\[
Y=\begin{cases}
Y^{*}, & R=1,\\
\textrm{NA}, & R=0.
\end{cases}
\]
Here $R$ is allowed to depend on $X$, with \emph{known} conditional
probability $\pi(x)=p(r=1\mid x)$. The joint density of $(X,Y,R$) is
\begin{eqnarray*}
p(x,y,r\mid\theta) & = & \begin{cases}
p(y\mid x,\theta)\pi(x), & r=1,\\{}
[1-\pi(x)]1\{y=\textrm{NA}\} & r=0.
\end{cases}
\end{eqnarray*}
Our goal is to estimate $\psi=P(Y_{i}^{*}=1)=\int p(y\mid x,\theta)dx$.
We do not wish to assume any specific parametric form for the true
$p(y\mid x,\theta)$. Anyhow, the posterior is proportional to
\begin{eqnarray*}
p(\theta\mid\textrm{data}) & \propto & \prod_{r_{i}=1}p(y\mid x_{i},\theta)\pi(x_{i})\prod_{r_{i}=0}1\{y=\textrm{NA}\}(1-\pi(x)),\\
 & \propto & \prod_{r_{i}=1}\pi(x_{i})\prod_{r_{i}=1,}p(y\mid x_{i},\theta).\\
 & \propto & \prod_{r_{i}=1,}p(y_{i}\mid x_{i},\theta).
\end{eqnarray*}
Hence the posterior does not depend on $\pi(x)$ unless the prior
does. However, \textcite{Robins1997-uv} proved that any uniformly consistent
estimator of $\psi$ must depend on $\pi(x)$. In this sense, any Bayesian estimator of $\hat{\theta}$ is inadequate. No matter what likelihood and prior you choose, the estimator of $\psi$ won't be uniformly consistent.

On the other hand, there is a well-known uniformly consistent estimator of $\psi$,
namely the \emph{Horwitz--Thompson} estimator
\begin{equation}
\hat{\psi}=\frac{1}{n}\sum_{i=1}^{n}\frac{Y_{i}R_{i}}{\pi(X_{i})}.\label{eq:Horwitz-Thompson}
\end{equation}
It is unbiased, and it variance shrinks to $0$ as $n\to\infty$,
hence it's consisent. Moreover, the interval $C=\{\hat{\psi}\pm([2n\delta^{2}]^{-1}\log(2/\alpha))^{1/2}\}$
is a valid finite sample level $\alpha$ confidence set, i.e.
\[
P(\psi\in C)\geq1-\alpha\quad(P\in\mathcal{P})
\]
for all $n$.
\end{example}

The Robins--Wasserman does not matter to a Bayesian that firmly knows the likelihood and the prior. For him, the Bayes' theorem is the best you could possible do. And by Doob's consistency theorem \parencite{Miller2018-xq}, if you know your prior is true, every Bayesian procedure is consistent.

The Bayesians that are troubled by this example are the bounded Bayesians, those who have only limited trust in their
priors and likelihoods, and try to make do with approximations. In
parametric problems, the bounded Bayesians are relatively safe. Due
to the Bernstein--von Mises phenomenon \parencite[Section 10.2]{Van_der_Vaart2000-qc},
the prior does not matter in the limit, as $n\to\infty$. Moreover,
parameter estimates such the posterior mean will converge to the do
the Kullback--Leibler-minimizing parameter \parencite[Theorem 2.1]{Bunke1998-vg},
sometimes called least-false parameters \parencite[p. 25]{Claeskens2008-hk},
just as they do for maximum likelihood estimation.

But non-parametric Bayesians doesn't have the luxury of an unconstrained
Bernstein--von Mises phenomenon. The imaginary logical omniscience
of the Bayesian has practical consequences now. In the Robins--Wasserman
example, the frequentist is luckier than the Bayesian, as he does
not have pretend to be omniscient at all. He does not need to pretend
to know the likelihood $p(y\mid x,\theta)$, much less a prior for
it. The Horwitz--Thompson estimator is an empirical average, not
a function of the likelihood.

As a Bayesian who admits not to be logically omniscient, what should
you do? The frequentist method suggests a possible solution. We are
looking for an esimator of the mean $\psi$, but the theoretical Bayesian
is forced to supply a likelihood for the far more general $\theta$.
But we can reformulate the problem so $\theta$ does not appear at
all. 
\begin{example}[Response to Robins--Wasserman]
 The bounded Bayesian will not know the likelihood of $\theta$,
but there is still hope for him, as he has some information about
the problem. In particular, he knows that $E[Y_{i}R_{i}/\pi(X_{i})]=\psi$.
As any statistician worth his salt, he pretends that $Y_{i}R_{i}/\pi(X_{i})$
is normal. He can then use e.g. a normal prior for the mean parameter
of this normal distribution, and analyze it as an ordinary normal-likelihood
and normal-prayer Bayesian problem.
\end{example}

\textcite{Sims2012-ze} proposes other bounded Bayesian solutions
to the Robins--Ritov--Wasserman problem. Wasserman and Robins interprets
such solutions as engaging in \emph{frequentist persuit}. That is,
an attempt to manipulate Bayesian procedures to have good frequentist
properties through \emph{ad hoc} metods. But the bounded Bayesian
does not do his analysis due to frequentist persuit. He does it because
he is aware of his own ignorance: No one can force you to know something
you don't, and you don't know the likelihood of $y\mid x$, much less
its prior. The bounded Bayesian retains some of his upper hand vis-à-vis
the frequentist, as he is able to incorporate his knowledge about
$\psi$ into his prior and has guaranteed frequentist properties via
the Bernstein--von Mises theorem.

That said, plenty of self-declared Bayesians are happy to engange
in frequentist persuit. In this section, ``Bayesian'' is used in
the sense of Finetti and Savage, as an ideal, rational actor. Most
Bayesians are not dogmatic in this sense, but pragmatic Bayesians
who use the methods because they often work better than frequentist
methods. For instance, \textcite{Gelman2013-ib} argue that the Bernstein--von Mises
phenomenon is essential to the Bayesian because it guarantees good frequentist properites of confidence sets. (On the other hand, \textcite{Morey2016-ry} argue that confidence sets are worthless unless they can be given a Bayesian interpretation.)

\subsection{Loss functions}

Loss functions don't receive much attention at all. Most statisticians
let $l$ be quadratic, the action space be the parameter space, and
call it a day. While this procedure certainly is practical, it's hard
to defend philosophically. Loss functions can be categorized into
at least five classes, depending on their actions space.
\begin{lyxlist}{00.00.0000}
\item [{\textbf{Point~prediction}}] Our goal is to predict the value of
a new sample of $X$. In this case, the action space $\mathcal{A}$
has to be a subset of $\dom X$, the domain of $X$. The loss function
is $l(a,x)=E_{X\mid\theta}[l'(a,x)]$, where $l'(a,x)$ should equal
$0$ if and only if $a=x$, and $l'(a,x)>0$ for all other $a\in\dom X$.
Well-known prediction losses of this kind are the $L_{p}$-losses,
$l'(a,x)=|a-x|^{p}$, the $0-1$-loss $l'(a,x)=1[a=x]$, the Linex
loss $l'(a,x)=b(e^{c(x-a)}-c(x-a)-1)$ \parencite{Varian1975-cd}, and
the Huber loss from robust statistics \parencite{Huber1964-fm}. For all
of these cases, the risk $R=\min_{a\in\mathcal{A}}E[l(a,x)]$ tells
you how certain you can be about your prediction of $X$.
\item [{\textbf{Density~prediction}}] The action space $\mathcal{A}$
is a subspace of the space of densities on $\mathcal{X}$. Again,
we should demand that $l(g,f)=0$ if and only if $g=f$ a.e., and
$l(g,f)>0$ for all other $g$. That is, $l$ is a statistical divergence
function. Some well-known divergences are the Kullback--Leibler divergence,
the $L_{2}$-divergence, the Hellinger distance, and the class of
$f$-divergences \parencite{Basu2011-gs}. 
\item [{\textbf{Parameter~estimation}}] The action space $\mathcal{A}$
is a subset of the parameter space $\Theta$. The loss function will
have the same properties as in point prediction, i.e., the loss $l(a,\theta)$
should equal $0$ if and only if $a=x$, and $l(a,x)>0$. The by far
most popular loss is the quadratic loss. The context is usually stated
as ``parameter estimation'', without any more details about the
problem or a statement as to why you should be interested in the parameters
themselves.
\item [{\textbf{Instrumental~losses}}] Sometimes we design our loss functions
to help us reach our statistical goals. One application is the formulation
of stopping rules in sequential analysis \parencite{Brockwell2003-pd}.
A natural stopping rule is to check the posterior risk, continue if
it is too large, and stop if it is too small. We add an increasing
penalty $C_{k}$ to the loss, where $k$ is the sample size, since
sampling costs money. The action space is the product two spaces.
The first space contains two actions, Stop and Continue. If the action
chosen is Stop, we will stop our sampling process; if it is Continue,
we will continue the sampling process. The second space is $\dom X$,
and the risk is $E_{X\mid\textrm{data}}[l(\mu,X)]+C_{k}$. Similar
instrumental losses can be used for e.g. sparse estimation.
\item [{\textbf{Practical~losses}}] Much decision theory is framed in
terms of practical losses but they are rarely used. A practical loss
is simply a loss that doesn't fit into the categories above. For example,
\textcite{Stankovic1985-th} employs custom-made loss for the problem
of decentralized control of job scheduling. 
\end{lyxlist}
The parameter estimation losses are not well-motivated. It's hard
to imagine any situation where anyone would directly care about the
posterior mean of $\mu$, philosophically speaking. Yes, posterior
means and standard deviations are easy to calculate and intertrep,
but this is hardly a deep justification to employ them. Losses on the parameter space does not take into account that parameterizations are, in a sense, arbitrary. For any identified $k$-dimensional parameterization of a likelihood, there are $2^{\mathfrak{c}}$ (where $\mathfrak{c}$ is the cardinality of the continuum) identified $k$-dimensional parameterizations, and why care only about one of them? 

For a better motivated variant of parameter estimation, we could re-frame the problem as a density estimation problem. For if let let the action space $\mathcal{A}$ be the space of likelihoods, we are essentially dealing with a \emph{parameterization-agnostic} estimation problem. 
\begin{example}
\label{exa:exponential estimation} Let $X_{1},\ldots,X_{n}\sim\Exp(\lambda)$
and $\lambda\sim\GammaDist(\alpha_{0},\beta_{0})$. It is well-known
that the posterior distribution is $\GammaDist(\alpha,\beta)$, with
parameters $\alpha=\alpha_{0}+n$, $\beta=\beta_{0}+n\overline{x}$.
The posterior predictive distribution is $\Lomax(\alpha,\beta)$,
which has density $\alpha\beta^{\alpha}(x+\beta)^{-\alpha-1}$. 

A popular way to do density prediction is to use the the posterior
predictive density. This\emph{ }is the point-wise solution to the
risk-minization problem $\min_{p}E_{\theta\mid\textrm{data}}\{[p(x)-f_{\theta}(x)]^{2}\}$,
where $f_{\theta}(x)$ is the aleatoric density at a data point. But
this density is not decision-theoretically justified; it is merely
a marginalized density. To justify the usage of the posterior predictive
density, we need a loss function over over the space of densities, as described above.
The best-known and most popular such loss function is the Kullback--Leibler\emph{
}divergence \parencite{Kullback1951-kv}
\begin{equation}
d_{KL}(f,g)=\int\log\left[\frac{f(x)}{g(x)}\right]f(x)dx.\label{eq:Kullback-Leibler}
\end{equation}
 Define the \emph{entropy} $H(f)=\int-\log[f(x)]f(x)dx$ and the \emph{cross-entropy}
$H(f,g)=-\int\log[g(x)]f(x)dx$. Assuming finite entropy and cross-entropy,
\begin{equation}
d_{KL}(f,g)=H(f,g)-H(f).\label{eq:Kullback--Leibler(entropy)}
\end{equation}
Provided $E_{\theta\mid\textrm{ data}}[H(f_{\theta})]$ is finite,
the posterior predictive distribution $\tilde{p}$ solves $\min_{p}E_{\theta\mid\textrm{data}}[d_{KL}(f_{\theta},p)]$.
That is, $\tilde{p}=\argmin_{p\in\mathcal{P}}E_{\theta\mid\textrm{data}}[d_{KL}(f_{\theta},p)],$
where $\mathcal{P}$ is the space of all densities. To see why, observe
that
\begin{eqnarray}
 &  & E_{\theta\mid\textrm{data}}\left(\int[\log f_{\theta}(x)-\log(g)]f_{\theta}(x)dx\right)\nonumber \\  
 & = & E_{\theta\mid\textrm{data}}[H(f_{\theta},g)]-E_{\theta\mid\textrm{data}}[H(f_{\theta})],\label{eq:cross-entropy equality}\\
 & = & H(\tilde{p},g)-E_{\theta\mid\textrm{data}}[H(f_{\theta})],\nonumber 
\end{eqnarray}
and that this is minimized in $g=\tilde{p}$, as the true distribution
minimizes the cross-entropy. 

Parameter estimation can be framed in similar terms. Estimating the
parameters of a density $f$ is just a rephrasing of estimating the
density. That is, estimating the $\lambda$ parameter in an
exponential model is the same kind of problem as estimating $f_{\lambda}$,
the exponential density itself. Now let $\mathcal{P}$ the family
of exponential densities . Then $\argmin_{p\in\mathcal{P}}E_{\theta\mid\textrm{data}}[d_{KL}(f_{\theta},p)]$,
an estimate of the density $f_{\lambda}$, can readily be calculated.
This method of estimating parameters is studied by \textcite{Robert1996-ii},
who calls the statistical divergences intrinsic losses. 

From the equations (\ref{eq:cross-entropy equality}) above, we know
we only have to minimize the cross-entropy $H(\tilde{p},g)$. Let
$X\sim\Lomax(\alpha,\beta)$, then calculate
\begin{eqnarray*}
H(\tilde{p},g_{\lambda}) & = & \int[\log\lambda-\lambda x]\alpha\beta^{\alpha}(x+\beta)^{-\alpha-1}dx,\\
 & = & \log\lambda-\lambda EX=\log\lambda-\lambda\frac{\beta}{\alpha-1}.
\end{eqnarray*}
This function reaches its minimum in $\hat{\lambda}_{KL}=(\alpha-1)/\beta$,
where $H(\tilde{p},g_{\lambda})=\log\beta-\log(\alpha-1)$. Compare
this to the posterior mean, which for this particular parameterization
is the slightly larger $\alpha/\beta$. 

Recall that maximum likelihood estimation can be formulated in terms
of Kullback--Leibler minimization \parencite[p. 25]{Claeskens2008-hk}:
If $F_{n}$ is the the empirical cumulative distribution function,
the maximim likelihood estimator is $\hat{\theta}_{ML}=\argmin d_{KL}(F_{n},f_{\theta})$.
Since $\hat{\theta}_{ML}$ is invariant under reparameterizations
(just like other minimum distance estimators, see \textcite{Drossos1980-ar}),
it is hardly surprising that the Kullback--Leibler loss minimizer
is invariant under reparameterizations too. 
\end{example}


\subsection{Frequentist statistics}

Frequentist inference is hard to understand for lay people. Misunderstandings
about \emph{p}-values and confidence intervals are ubiquitous, documented
in fields such as psychology \parencite{Belia2005-di,Gigerenzer2018-oi}
and medicine \parencite{Goodman2008-ed,Gigerenzer2007-qi}. Frequentist quantities are almost never unique in the same
sense as Bayesian quantities, and they are often hard to reason about
mathematically. 

The much harder mathematics involved in research level frequentist
statistics has led to an understanding that Bayesian statistics is
boring. 
\begin{quote}
What is the principal distinction between Bayesian and classical statistics?
It is that Bayesian statistics is fundamentally boring. There is so
little to do: just specify the model and the prior, and turn the Bayesian
handle. There is no room for clever tricks or an alphabetic cornucopia
of definitions and optimality criteria. I have heard people who should
know better use this 'dullness' as an argument against Bayesianism.
One might as well complain that Newton's dynamics, being based on
three simple laws of motion and one of gravitation, is a poor substitute
for the richness of Ptolemy's epicyclic system. \parencite{Dawid2000-aw}
\end{quote}
It's easy to forget just how strange hypothesis tests, confidence
sets, and \emph{p}-values are. 

\subsubsection{Hypothesis tests, \emph{p}-values, and confidence intervals}

In the following pages, $(\Omega,\mathcal{\mathcal{F}})$ will be
a measurable space and $\mathcal{P}$ a backgroud family of probability
measures on this space. The family $\mathcal{P}$ contains every probability
measure we consider plausible. 
\begin{definition}
\parencite[][Chapter 3.1]{Lehmann2005-sp} Let $\mathcal{P}_{0}$ be a family
of probability measures on $(\Omega,\mathcal{F})$. A test of the
null-hypothesis $P\in\mathcal{P}_{0}$ of \emph{size} $\alpha$ is
a set $R$ such that $\sup_{P\in\mathcal{P}_{0}}P(R)=\alpha.$ A test
of $P\in\mathcal{P}_{0}$ of \emph{level $\alpha$ }is a set $R$
such that $\sup_{P\in\mathcal{P}_{0}}P(R)\leq\alpha.$
\end{definition}

The set $R$ is the \emph{rejection set} of the hypothesis test. Its
complement is the acceptance set of the hypothesis and is denoted
$A=R^{c}$. The philsophical underpinning of hypothesis test, due
to Neymann, is that you have to have make a binary choice. Either
you act as if $P\in\mathcal{P}_{0}$ is true, or you act as if $\mathcal{P}_{0}$
isn't true. When you do a hypothesis test, you choose $\mathcal{P}_{0}$
if $\omega\in R^{c}$ and $\mathcal{P}_{1}=\mathcal{P\backslash P}_{0}$,
the alternative hypothesis, otherwise. The definition of hypothesis
test guarantees you will choose $\mathcal{P}_{1}$ when
$\mathcal{P}_{0}$ is true with at most probability $\alpha$. Usually,
discussions of hypothesis tests will involve the probability that
$\mathcal{P}_{1}$ is chosen when $\mathcal{P}_{1}$ is true; this is called the
\emph{power} of the test \parencite{Neyman1977-nx}.

Hypothesis tests are quite easy to understand, especially when formulated
with explicit null-hypotheses and alternative hypotheses. The complexity
goes up a notch with \emph{p-}values.
\begin{definition}
\label{def:p-value}(\textcite[][Chapter 3.3]{Lehmann2005-sp}, \textcite{Bayarri2000-dt}) Let
$A\subseteq[0,1]$ and $R(\alpha)$ be an increasing sequence of size
$\alpha$ rejection sets under $\mathcal{P}_{0}$, i.e., $R(\alpha')\subseteq R(\alpha)$
when $\alpha'\leq\alpha$, and $\sup_{P\in\mathcal{P}_{0}}P(R)=\alpha$.
Then random variable
\begin{equation}
U(\omega)=\inf\{\alpha\mid\omega\in R(\alpha)\}\label{eq:size p-value}
\end{equation}
is a \emph{p}-value. 
\end{definition}

Observe that $U\leq\alpha=\left\{ \omega\mid\inf\{\alpha'\mid\omega\in R(\alpha')\}\leq\alpha\right\} =R(\alpha)$.
Importantly, $U$ satisfies $\sup_{P\in\mathcal{P}_{0}}P(U\leq\alpha)=\sup_{P\in\mathcal{P}_{0}}P(R(\alpha))=\alpha$
for all $\alpha\in A$. When $\mathcal{P}_{0}$ is a singleton, $P(U\leq\alpha)=P(R(\alpha))=\alpha$.
In particular, if $A=[0,1]$, $U$ is uniformly distributed under
$P$, a common definition of a \emph{p}-value in and of itself. The definition
of \emph{p}-value is slightly more general than usual, as it allows
for both $A\neq[0,1]$ and composite hypotheses. The \emph{p}-value
suffers from an all-too-common problem in statistics; there is no
different name for the observed \emph{p}-value $u$ and the \emph{p}-value
statistic $U$. \textcite{Schweder1988-nh} proposed to call $U$ the
``signficance statistic'', a name that has unfortunately not caught
on. 

You could claim this definition of a \emph{p}-value is too convoluted.
But common definitions of\emph{ p}-values are incomplete, most of
them being variants of ``the probability of observing something at
least as extreme as the observed data, given the null hypothesis is
true.'' A better definition is ``the probability of observing $T\geq t$,
where $t$ is a observation of the statistic $T$'', as it makes
the dependence on the (often arbitrary) statistic $T$ explicit. But
stating it in terms of of increasing rejection sets is even better,
as it makes it clear just how many \emph{p}-values there are, how
permissive the definition is, and how the definition is fundmentally
about chains in the $\sigma$-algebra $\mathcal{F}$. Moreover, the
connection between hypothesis tests and \emph{p}-values is easiest to state
and appreciate in terms of rejection sets. The notion of most powerful
\emph{p}-value is obvious; a \emph{p}-value is most powerful against
$\mathcal{P}_{1}$ if each rejection set $R(\alpha)$ is a most powerful
size $\alpha$ hypothesis test. The usual formulation hides this though, stating only $H_0:\mu = 0$, not $H_0:\mu=0,\sigma>0$.

To see why \emph{p}-values should be defined for composite hypothesis,
consider the most famous of test of them all, the two-sided $t$-test.
The null-hypothesis $\mathcal{P}_{0}$ is the family of normal probabilities
with mean zero and any standard deviation $\sigma$, which is composite.
Luckily, $\sqrt{n}\overline{x}/s$ is a pivot in this situation, i.e.,
$P_{\sigma}(\sqrt{n}\overline{x}/s\leq x)$ is independent of $\sigma$,
but the null-hypothesis is still composite. 

How should you use \emph{p}-values for inferential purposes? That's
really hard to say. First of all, notice that a \emph{p}-value does
not have the error-rate interpretation as an hypothesis test. A of
p-value of $0.05$ is not observed with probability $0.05$ under
the null-hypothesis, but a \emph{p}-value of $0.05$ or less is observed
with probability $0.05$. 

It is hard to interpret \emph{p}-values. The definition is opaque and hard, if not impossible, to connect to real-life outcomes. And it is hard to justify the use numbers one cannot expect anyone to interpret in any meaningful way, especially because people will try to interpret it, and inevitably fail. \textcite{Cohen1994-au} wrote, in his critique of \emph{p}-values, 
\begin{quotation}
What's wrong with {[}null-hypothesis significance testing{]}? Well,
among many other things, it does not tell us what we want to know,
and we so much want to know what we want to know that, out of desperation,
we nevertheless believe that it does! What we want to know is \textquotedbl Given
these data, what is the probability that $H_{0}$ is true?\textquotedbl{}
\end{quotation}
Some authors, most famously Fisher \parencite{Liu2020-er} attempt to justify \emph{p}-values
as measures of evidence, and, according to \textcite{Berger1987-tf}
most statisticians use \emph{p}-values since they are ``feeling it
to be important to indicate how strong the evidence against $H_{0}$''.
\textcite{Hubbard2008-cg}, among others, are strongly critical of the
idea that \emph{p}-values are a measure of evidence, but \textcite{Liu2020-er} defend a slight modification of Definition \ref{def:p-value}, incorporating asymptotic guarantees, as being a reasonable measure of evidence.

One reason why \emph{p}-values are not measures is evidence is the lack of explicit alternative hypotheses. That \emph{p}-values are usually framed without explicit alternative hypotheses is sometimes framed as a strength. \textcite[p. 308]{Barnard1962-rz} wrote that ``the simple tests of significance arise ,it seems to me, in situations where we do not have a parameter space of hypotheses; we have only a single hypothesis essentially, and the sample space then is the only space of variables present in the problem.`` Fisher was firmly against formal alternatives hypotheses and power calculations \parencite{Lehmann1993-oa}. A compelling argument for why alternative hypotheses are important is the Albino argument of \textcite{Berkson1942-hj}:
\begin{quotation}
Suppose I said, \textquotedblleft Albinos are very rare in human populations, only one in fifty thousand. Therefore, if you have taken a random sample of 100 from a population and found in it an albino, the population is not human.\textquotedblright{} This is a similar argument but if it were given, I believe the rational retort would be, \textquotedblleft If the population is not human, what is it?\textquotedblright{} 
\end{quotation}
A major difference between the Neyman--Pearson theory of statistical
tests and \textit{p}-values is the theoretical justification. Classical Neyman--Pearson-type
inferential statistics, with Lehmann's ``Testing statistical
hypotheses'' \parencite{Lehmann2005-sp} as its bible, is concerned with
finding optimal tests. Usually, uniformly most powerful tests or uniformly
most powerful unbiased tests do not exist, but it is sometimes possible
to find other kinds of optimal tests. Optimality theory gives you
justification for selecting one test instead of another, and maybe
more importantly, forces you to think clearly about exactly what your
assumptions are and exactly what you want to test. \emph{p}-values,
almost never framed in this way, are much more arbitrary. 

Confidence sets are even harder handle than \emph{p}-values. For while
\emph{p}-values are defined in terms of a nested family of rejection
sets, a confidence set needs a parameterized family of rejection sets.
\begin{definition}
\label{def:confidence sets}Let $\Pi=\{\mathcal{P}(\theta)\}_{\theta\in\Theta}$
be a partition of probabilities on a common probability space $(\Omega,\mathcal{F})$.
A confidence set of level $\alpha$ is mapping $R:\Theta\to\mathcal{F}$,
a family of rejection sets, satisfying 
\begin{equation}
\sup_{\theta\in\Theta}\sup_{P\in\mathcal{P}(\theta)}P(R_{\theta})\leq\alpha\label{eq:confidence set}
\end{equation}
If the inequality is an equality, the confidence set has size $\alpha$. 
\end{definition}

Usually we define a set $C(\omega)$ by $\theta\in C(\omega)\iff\omega\notin R_{\theta}$ and call $C$ a confidence set. The definition above uses the well-known duality between rejection sets and confidence sets, and constructing a confidence set from rejection sets in this way is called
\emph{inverting a test}. The definition might look more abstract than necessary, as it involves two suprema. But both are necessary, despite they not showing up in the most familiar examples. The supremum over $\mathcal{P}(\theta)$ guarantees that that $P(R_{\theta})\leq\alpha$
for every $P$ in the equivalence class $\mathcal{P}(\theta)$, while
the supremum over $\Theta$ guarantees that this is the case for each equivalence class.

A confidence set should be regarded as family of rejection sets, not
as a random set. For it to be random set we would have to find a $\sigma$-algebra
$\mathcal{G}$ over a suitable space of sets $\mathcal{X}$ such that
$C:\Omega\to\mathcal{X}$ is measurable, i.e., $C^{-1}(G)\in\mathcal{F}$
for every $G\in\mathcal{G}$. Luckily, this is unneccessary, as we
are never interested in questions such a ``What is the probability
under $P$ that $C$ is an element of $\mathcal{D}$?''. But a formalization
of $C$ as a random set would would be necessary to answer such questions.

Confidence sets are not only hard to define, they are hard to understand.
You can't figure a handy reformulation of confidence sets, as that
does not exist \parencite{Morey2016-ry}. 
\begin{example}[{\textcite[Example 4a]{Berger1988-ji}}]
 Let $X\in\{1,2,3\}$ and $\theta\in\{1,2\}$. Define

\[
P_{0}(x)=\begin{cases}
0.009, & x=1.\\
0.001, & x=2,\\
0.99, & x=3,
\end{cases},\;P_{1}=\begin{cases}
0.001, & x=1.\\
0.989, & x=2,\\
0.01, & x=3,
\end{cases}
\]
The rejection set $R=\{x\neq3\}$ is the most powerful test of $P_{0}$
vs $P_{1}$, with both error probabilities equal to $0.01$. Now suppose
you observe $x=1$, whereupon you would reject $P_{0}$ in favour
$P_{1}$ according to $R$. But $x=1$ is $9$ times more likely under
$P_{0}$ than under $P_{1}$!
\end{example}

Moreover, confidence sets can be empty; even optimal confidence sets
can be empty \parencite{Blaker2000-ud}. Confidence set for a real parameter
can contain the entire real line \parencite{Morey2016-ry}. Most confidence
sets are asymptotic, but with no guaranteed coverage since uniform
convergence almost never is proved, or even true \parencite{Gleser1996-kk}.

\subsubsection{Impossibility results}

Let's say you have a sequence of real random variables $X_{1},X_{2},\ldots,X_{n}$
from some probability measure $P$ with finite mean $\mu=EX_{1}<\infty$.
You do not know anything else about $P$. Is it possible to say anything
about the mean $\mu$? From a Bayesian perspective the answer is no,
as the problem is underspecified. The problem lacks not only a prior,
but a likelihood too. Even worse, there is no $\sigma$-finite dominating
measure for the class of candidate $F$s, so Bayes' theorem would
be useless in any case. 

But what happens with \emph{p}-values? It turns out there is no nice
\emph{p}-value for the mean, or even hypothesis test, for non-parametric
families of distributions, even when the variance is finite for every
$P$. \textcite{Bahadur1956-tg} proved the following elegant theorem.
\begin{theorem}[Bahadur--Savage]
Let $\mathcal{\mathcal{P}}$ be a family of probability measures
satisfying
\begin{enumerate}
\item[i.)] The expectation $\mu_{F}=\int xdP$ exists and is finite for every
$P\in\mathcal{P}$.
\item[ii.)] For every $\mu\in\mathbb{R}$, there is an $P\in\mathcal{\mathcal{P}}$
satisfying $\mu(P)=\mu$.
\item[iii.)] The family $\mathcal{\mathcal{P}}$ is closed under convex combinations.
That is, if $\lambda\in[0,1]$ and $P,Q\in\mathcal{P}$, then $\lambda P+(1-\lambda)Q\in\mathcal{P}$
too.
\end{enumerate}
Let $\mathcal{P}_{n}(\mu)$ denote the family of probability measures
containing every $P^{n}$ such that $P\in\mathcal{P}$ and $\mu(P)=\mu$.
Let $R$ be a size $\alpha>0$ rejection set for the hypothesis $\mu(P)=\mu_{0}$.
Then 
\[
\sup_{P\in\mathcal{P}_{n}(\mu)}P(R)=\alpha,\quad(\mu\in\mathbb{R})
\]
that is, the maximal probability of rejecting $H_{0}:\mu(P)=\mu_{0}$
is the same for every equivalence class of distributions. That is,
the test has no unifom power against any alternative hypothesis. 
\end{theorem}

\begin{proof}
Let $R$ be a size $\alpha$ rejection set, in other words, $\sup_{P\in\mathcal{P}_{n}(\mu_{0})}P(R)=\alpha$,
and choose a $\mu\neq\mu_{0}$. Let $\epsilon>0$ be arbitrary and
select a witness $P_{0}\in\mathcal{P}_{n}(\mu_{0})$ satisfying $\alpha\ge P_{0}(R)>\alpha-\epsilon$.
Choose a pair $\lambda,\eta$ such that $\mu=(1-\lambda)\mu_{0}+\lambda\eta$.
Let $P_{\eta}(R)\in\mathcal{P}_{n}(\eta)$ and define $P_{\mu}$ by
$P_{\mu}(A)=(1-\lambda)P_{\mu_{0}}(A)+\lambda P_{\eta}(A).$ When
$A=R$, we obtain the inequality
\begin{eqnarray*}
P_{\mu}(R) & = & (1-\lambda)P_{\mu_{0}}(R)+\lambda P_{\eta}(R),\\
 & \leq & (1-\lambda)\alpha+\lambda.
\end{eqnarray*}
Letting $\lambda\to0$ along with $\eta$, we see that $\sup P_{\mu}(R)\leq\alpha$.
On the other hand,
\begin{eqnarray*}
P_{\mu}(R) & = & (1-\lambda)P_{\mu_{0}}(R)+\lambda P_{\eta}(R),\\
 & \geq & (1-\lambda)(\alpha-\epsilon),
\end{eqnarray*}
hence $\sup P_{\mu}(R)\geq\alpha-\epsilon$ by letting $\lambda\to0$.
Since this is true for every $\epsilon,$ the result follows.
\end{proof}
The Bahadur--Savage theorem tells that for any sample size $n$,
there is no hypothesis test of the mean that has power agianst a whole
class of distributions with any other mean. In a sense, the mean is
untestable.

% Assume $\{P_{\mu}\}_{\mu\in\mathbb{R}}$ is a family generating family
% for $\mathcal{P}$, so that every $P\in\mathcal{P}$ has the representation
% \[
% P=\sum_{i=1}^{n}\lambda_{i}P_{i},\quad(\sum_{i=1}^{n}\lambda_{i}=1,\,P_{i}\in\mathcal{P},\lambda_{i}\in(0,1)).
% \]
% For instance, each $P_{\mu}$ could be the probability distribution
% of a normal variable with mean $\mu$ and standard deviation $\sigma_{i}=1$.
% All of these mixture components are assumed to have finite variance.
% Using the formula for the variance of a mixture, $\Var P=\sum_{i=1}^{k}\lambda_{i}(\sigma_{i}^{2}+\mu_{i}^{2}-\mu^{2})$,
% where $\mu=\sum_{i=1}^{n}\lambda_{i}\mu_{i}$ is the mean of $P$.
% Thus the variance of each $P$ is finite. 

% Now consider the studentized mean

% \begin{equation}
% \sqrt{n}\frac{\overline{x}-\mu}{s}\stackrel{d}{\to}N(0,1).\label{eq:studentized mean}
% \end{equation}
% where $\overline{x}$ is the sample mean and $s$ is the sample standard
% deviation. By the Bahadur--Savage theorem, the natural rejection
% set $R_{\mu}=\{x\mid\sqrt{n}|\overline{x}-\mu|/s\geq z_{\alpha/2}\}$
% is has no power against any any alternative, and is useless.

A more recent example of utestable hypothesis testing is \emph{conditional
independence testing}. Consider a density with three variables $x,y,z$
and the following test
\begin{align*}
\mathcal{P}_{0}: & p(x,y\mid z)=p(x\mid z)p(y\mid z)\quad\textrm{for all }x,y,z.\\
\mathcal{P}_{1}: & p(x,y\mid z)\neq p(x\mid z)p(y\mid z)\quad\textrm{for some }x,y,z.
\end{align*}
\textcite{Shah2018-jh} proved the following theorem, showing that a
conditional independence test has no power against any alternative. 
\begin{theorem}
\label{theorem:Shah--Peters}Let $n$ be arbitrary and $x_{i},y_{i},z_{i}$
be identically and independently sampled from $P$. Assume $P^{n}(R)\leq\alpha$
for all $P\in\mathcal{P}_{0}$. Then $P^{n}(R)\leq\alpha$ for all
$P\in\mathcal{P}_{1}$ too.
\end{theorem}

The Shah--Peters theorem is stronger than the Bahadur--Savage theorem
in the sense its alternatives are simple, not composite. The alternatives
in Bahadur--Savage are entire classes of probability measures; and
the richness of these classes is what drives the theorem. The Shah--Peters
theorem, on the other hand, is about the richness of the the nullhypothesis,
$\mathcal{P}_{0}$. This class of distributions is so large that any
rejection set satisfying $P^{n}(R)\leq\alpha$ for all $P\in\mathcal{P}_{0}$
must satisfy $P^{n}(R)\leq\alpha$ for all $P\in\mathcal{P}_{1}$
too.

An analogue of the Shah--Peters result does not hold for independence
testing, as the rank test of \textcite{Hoeffding1948-nm} test is pointwise
consistent. Still, there is no uniformly unbiased test of independence \parencite{Moss2020-bc}.
