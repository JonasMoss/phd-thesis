\section{Introduction}

You are reading a statistics thesis about statistics in psychology. Statistics in psychology comes in two packages, and this thesis touches both them.

\textbf{Psychometrics.} What is intelligence? How do we measure it? How many personality traits are there, and how do they matter? Psychometrics is the science of psychological measurement. There are two kinds of psychometricians; the theoretical and practical. Theoretical psychometricians are just like statisticians, they deal with mathematics and programming. They publish in journals such as \textit{Psychometrika}, \textit{British Journals of Mathematical Psychology}, and \textit{Multivariate Behavioural Research}. These journals are specialized methodological journals, and allow for the use of mathematics. Practical psychometrians design measurement instruments and administer them. They publish both in specialized psychometric journals such \textit{Psychological Assessment}, and methodologically generalist psychology journals such as \textit{Emotion.} The section on psychometrics shows the main models of psychometrics, discusses some fundamental questions, and gives an intuition for what questions we are dealing with.

\textbf{Psychological~methods.} Psychological methodology is much wider in scope, encompassing everything that has anything to do with psychological research. Should you use frequentist hypothesis tests, Bayes factors, or maybe avoid testing altogether? When is Pearson's correlation coefficient more appropriate than Spearman's rho? When would multilevel modeling be appropriate? How can you do a high-quality meta-analysis? Pretty much of all applied statistics is relevant to psychologists in some way, and research in statistics is published in psychology journals such as the generalist journals \textit{Psychological Bulletin} and \textit{Psychonomic Bulletin Review}, and the psychological methodology journals \textit{Advances in Methods and Practices in Psychological Science}, and \textit{Psychological Methods}. Psychology is undergoing a replication crisis, and has done so since about $2011$. That was the year when \cite{Bem2011-vq} published his study on paranormal stuff in the top-tier \textit{Journal of Social and Personality Psychology} and \cite{simmons_false-positive_2011} published their famous \textit{False-Positive Psychology} paper. The issue can be summed up like this: You can't trust what psychologists say. Partly due to the replication crisis, psychology is undergoing a reproducibility revolution. The section open science is about what this entails for psychology specifically, and the role of $\mathtt{R}$
and how open science should affect the field of statistics. 

Aside from the psychological context and open science, this thesis discusses two issues not every statistician is familiar with, namely impossibility results in frequentist estimation and partial identifiability. The section on statistical inference discusses some of the pros and cons of Bayesian and frequentist statistics, and will remind you about some definitions you might not use that much. Problems involving partial identifiability are everywhere. Whenever you hear someone say something like ``we let $\theta=1$ to make the problem identified,'' chances are you you are facing a tucked-away partial identification problem. And the final section of is about partial identifiable, illustrated with several examples. 